%\documentclass{report}
\documentclass[final,twoside,12pt]{article}
\usepackage{fullpage,a4wide}
\usepackage{listings}
\newcommand{\fixbugs}{{\emph{Fixbugs} }}
\newcommand{\java}{{\tt java} }
\newcommand{\replace}{{\tt replace ... with} }
%\newcommand{\vbar}{\|}

\lstset {
    language=java,
    numbers=left,
    numberstyle=\footnotesize,
    frame=single
}

\author{Richard Warburton\\richard.warburton@gmail.com}
\title{\fixbugs Tutorial}

\begin{document}
\maketitle
\tableofcontents
\newpage

\section{Introduction}
\fixbugs is a program for applying automated transformations to source level
\java programs.  It uses a domain specific language in order to describe these
transformations.  This document is aimed for people who want to understand the
\fixbugs language and use it in their programming projects.  You should learn
how to specific your transformations and how to use the tool in order to automatically
apply these transformations.  The transformations can be used for a variety of 
tasks for example bug fixing, refactoring and semantic patches.  If you have a messy
legacy \java codebase we're here to help you out.

The source code for \fixbugs can be downloaded from its launchpad project at
https://launchpad.net/fixbugs.  In order to download the source code you will need the
{\tt bzr} revision control system.  The README file in the distribution explains
how to compile and run \fixbugs.

The irc channel is \#fixbugs on irc.oftc.net.  Questions are welcome.

\section{Pattern Matching}
Pattern Matching is fundamental to the nature of \fixbugs.  It is the basic tool with
which one transforms programs.  Our pattern matching construct - \replace - takes a statement
and replace it with another statement.  The patterns themselves are modelled after \java syntax,
since this is what we are transforming.  Note that all transformations are based on first
order abstract syntax, so whitespace doesn't matter.

Metavariables in \fixbugs match against fragments of abstract syntax, and can be referred to in
the replacement rule.  This way blocks of the program can be referred to in the replacement rules,
and patterns can be generalised easily.  Section \ref{sec:side-conditions} describes how they can be
used in order to make semantic comparisons using temporal logic and certain predicates.

We begin with some simple example statements, the following removes any variable assignment:

\lstinputlisting{../contrib/removeVars.trans}

The {\tt ::t} statement at the beginning pattern matches any type as the meta-variable {\tt t}.
{\tt x} matches the name of the variable, whilst {\tt rhs} matches anything on the right hand side.
This is replaced with a {\tt ;} statement, ie a skip.  Later on we shall use more complex side conditions,
but this specification is applied {\tt ALWAYS}, thus its appearance at the end of the replacement statement.

We can narrow the matched statements by the type of the assignment, and also the expression on the right hand side.
We can also use the bound metavariables in constructing a new statement.  For example, here is a replacement rule
that substitutes any integer assignment by its left hand side:

\lstinputlisting{../contrib/remove_int_rhs.trans}

The general idea can be extrapolated to many different types of statement, and their associated syntactic patterns,
for example here we remove any {\tt for} statements that iterate on increments within a program:

\lstinputlisting{../contrib/remove_for.trans}

The {\tt x} and {\tt y} metavariables bind to the opening statements within the {\tt for} loop, whilst
{\tt i++} ensures that only loops that involve a syntactic increment are matched.  A new concept here
is that of the wildcard: {\tt ....} which matches any statement.  This is a useful construct that is
used here to ensure that any {\tt for} loops - either single statement or with a block for a body are matched.

block matching

multiple matches

\section{Side Conditions}
\label{sec:side-conditions}

In this section we shall be increasing the expressiveness of our specifications by using side conditions
that may enable or disable them.  The side conditions are written in a variant of temporal logic that is
also described.  So far you have only seen the keyword {\tt ALWAYS} at the end of your patterns, from now
on we shall be replacing it with {\tt WHERE} and a logical statement.  {\tt ALWAYS} is equivalent to
{\tt WHERE True}.

Another way in which we shall be extending basic patterns is by using labels.  Labels are a way of marking
a statement matched by a pattern.  These statement labels can then be referred to within a side condition.
In the following fragment the metavariable {\tt n} is bound to any assignment statement.

\begin{lstlisting}
    n: ::t x = rhs;
\end{lstlisting}

\subsection{Logic}

For the following couple of sections we shall be simply be talking about fragments of \fixbugs syntax that
are used as side conditions, ie after {\tt WHERE}.  You can see how these fit together into full specifications
in Section \ref{sec:side-egs}.  The building blocks of our side conditions are those of first order logic.
Consequently the logical constructs {\tt True}, {\tt False}, {\tt and}, {\tt or} and {\tt not} are used with their
usual semantics.  Here are some valid Fragments:

\begin{lstlisting}
    True
    not False
    True and False
    not (not False or not True)
\end{lstlisting}

In addition to these constructs we also allow a temporal condition, that can hold true at a specific statement.
This is specified by the following general form:

\begin{lstlisting}
    { temporal-condition } @ node
\end{lstlisting}

Here {\tt node} is a metavariable that has been pattern matched, for example by using a label.  We describe
temporal conditions henceforth.  This statement holds true $iff$ the temporal condition holds true at the
given node.

\subsection{Temporal Conditions}

NB: in order to differentiate between logical operators that can be used both in the temporal and non-temporal
conditions we choose to give them different syntax, rather than overload.  In the different situations they mean
subtley different things, but from most user's points of views that can be ignored.  Here is a table of the corresponding
terms:

\begin{eqnarray}
\nonumber {\tt Side Condition} & {\tt Temporal Condition} \\
\nonumber {\tt True} & {\tt true} \\
\nonumber {\tt False} & {\tt false} \\
\nonumber {\tt and} & \wedge \\
\nonumber {\tt or} & \mid \\
\nonumber {\tt not} & {\tt !}
\end{eqnarray}

We base our temporal logic on Computational Tree Logic, consequently our temporal-conditions hold true if a path
statement holds true over a path.  We quantify this by stating that they either hold true over all paths, or
that there exists some path where they hold true.  The syntax for these respectively is:

\begin{lstlisting}
    A [ path-condition ]
    E [ path-condition ]
\end{lstlisting}

There are then four path operators for describing how a node condition holds over a given path:

\begin{tabular}{|l|l|}
\hline
Fragment & Fragment holds true iff ... \\
\hline
{\tt X condition} & condition holds true in the next statement \\
{\tt F condition} & condition holds true at some point in the future \\
{\tt G condition} & condition holds true at every point in the future \\
{\tt condition1 U condition2} & condition1 holds true at every point until condition2 holds true \\
\hline
\end{tabular}
\\
\\
\\
Consequently the following are all valid temporal conditions:
\\
\begin{lstlisting}
    A [ X true ]
    E [ F false ]
    E [ G false | true ]
    A [ (false | true) U E [ X true ] ]
\end{lstlisting}

\subsection{Useful Predicates}

Its obviously quite useless to have all this temporal quantification if we only use {\tt true} and {\tt false} as
our base statements.  Consequently we also have several predicates that are built into \fixbugs and can be used
in logical conditions.  Predicates can either be local or global - local predicates are used within temporal conditions,
and global predicates within side conditions.

{\tt node} is a local predicate that takes one argument, that being a metavariable that binds to a node, and
holds true at the node and false otherwise.  Consequently "{\tt { node(x) } @ x}" is a tautology.

{\tt stmt} takes a statement pattern as an argument and holds true when the current statement matches that pattern.
It is a local predicate.  The statement patterns are identical to those used in {\tt REPLACE} statements.

{\tt type} is a predicate that takes two arguments: a string representing a type pattern and a metavariable that binds to
a variable name.  If the type of \java variable referred to by the metavariable matches the type pattern then the type
predicate holds.

{\tt method} is a predicate that allow one to pattern match the name of the method that surrounds the current node.  This
can be especially useful when working with interfaces such as {\tt Runnable} where there is a named method, {\tt run} which
has a specific meaning.

\subsection{Examples}
\label{sec:side-egs}

\section{Advanced Usage}

So far we have only covered the {\tt REPLACE} action - here we discuss strategies and also a few more advanced transformations.

\subsection{Strategies}

A Strategy is a operator that allows us to combine transformations.  A Strategy applied to transformations also becomes a
transformation, so they can be combined arbitrarily.

\subsubsection{DO}

The {\tt DO} Strategy has the following overall structure:

\begin{lstlisting}
DO
    transformation1
AND
    transformation2
AND
    transformation3
...
\end{lstlisting}

It sequentially applies {\tt transformation1} and then {\tt transformation2} and then {\tt transformation3} etc.
The minimum number of transformations to apply is two.  The variable bindings are disjoint between the transformations.
If One transformation fail to apply then its successors also fail.

\subsubsection{PICK}

The {\tt PICK} strategy is similar to the {\tt DO} strategy, except that instead of sequentially combing its inner
transformations, it provides a non-deterministic choice between its components.  This is useful for accounting for different
syntactic forms that some component within a transformation may take.  For example doing one thing on the main branch of
an {\tt if} statement, and another thing on the {\tt else} branch.

%\subsection{More Actions}

\section{Miscellaneous}

\subsection{Bug Reports}

Bug reports are welcome and can be filled online at https://bugs.launchpad.net/fixbugs.

\subsection{Contributing}

\fixbugs is written in a mixture of {\tt scala} and \java.
Patches can either be filed on the bug tracker, or emailed personally to me.  Additionally
a {\tt bzr} repository that can be publically pulled from is a perfectly acceptable approach
to submitting a patch.

% \section{Complete Syntax Pattern Listings}
% TODO

\end{document}

